\documentclass[onecolumn]{article}
\usepackage{graphicx} % Required for inserting images
\usepackage{amsmath}
\usepackage{amsfonts}
\usepackage{pythonhighlight}
\usepackage{datetime}
\usepackage[colorlinks]{hyperref}
\usepackage{titling}
\usepackage{matlab-prettifier}
\usepackage[a4paper, total={6in, 8in}]{geometry}

\makeatletter
\Hy@AtBeginDocument{%
  \def\@pdfborder{0 0 1}% Overrides border definition set with colorlinks=true
  \def\@pdfborderstyle{/S/U/W 1}% Overrides border style set with colorlinks=true
                                % Hyperlink border style will be underline of width 1pt
}
\makeatother

\hypersetup{%
  colorlinks=true,
  linkcolor=blue,
  linkbordercolor=blue,% 
}
\footskip = 1pt
\textheight = 700pt
\setlength{\droptitle}{-10em}

\title{Lab 2 Extra \\ \Large{IN3170 - Microelectronics}}
\author{Andreas Engøy, Erik Røset \& Daniel Tran}
\date{\monthname[\the\month] \the\year}

\begin{document}
\maketitle
\vspace*{50mm}
\tableofcontents

\section{Task 1}

\subsection{Objective}
The objective of this task is to build two CMOS inverter chains of different length and probing before the first and after the last inverter. The purpose of this is to measure and calculate the propagation delay a single inverter without knowing the added capacitance of the scope probe or directly measuring the input capacitance of just one inverter.

\subsection{Equipment}

\begin{table}[h]
    \centering
    \begin{tabular}{|c|c|c|}
        \hline
        \textbf{Component} & \textbf{Model} & \textbf{Quantity} \\
        \hline
        Hex Scmitt-Trigger Inverters & SN74HC1 & 1 \\
        Oscilloscope & HP54622 & 1 \\
        Waveform generator  & HP33120 & 1\\
        Voltage source & HPE3631 & 1 \\
        Breadboard & ~ & 1 \\
        \hline
    \end{tabular}
    \caption{List of components used in task 1.}
    \label{tab:bom}
\end{table}

\clearpage

\subsection{Method}

\begin{figure}[h!]
    \centering
    \includegraphics[width=0.5\textwidth]{Circuit_draw.jpg}
    \caption{Picture of the setup for task 1.}  
    \label{fig:circuit}
\end{figure}

\begin{figure}[h!]
    \centering
    \includegraphics[width=0.5\textwidth]{circuit_schematics.png}
    \caption{Schematic of the setup for task 1.}
    \label{fig:schematic}
\end{figure}

In this task we made two inverter chains, one with 3 inverters and one with 4 inverters. By using the IC SN74HC1 we only made 1 chain with 4 inverters, and then used the first 3 inverters to make the chain with 3 inverters. We probed the output of the third inverter with a oscilioscope to measure the propagation delay of the inverter chain with 3 inverters. We then probed the output of the fourth inverter to measure the propagation delay of the inverter chain with 4 inverters. The input was proved with channel 1 on the oscilioscope and both of the output was probed using channel 2 on the oscilioscope.

The input signal was a square wave with a frequency of 400Hz that was connected to pin 1, the input of first inverter in the chain. On pin 14, we connected a 5V power supply. 

\end{document}